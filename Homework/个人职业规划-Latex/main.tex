\documentclass{article}
\usepackage[UTF8]{ctex}
\usepackage{geometry}
\usepackage{multirow}
\usepackage{natbib}
\usepackage{graphicx}
\usepackage{setspace}
\usepackage{enumerate}
\usepackage{caption2}
\usepackage{datetime}

\pagestyle{plain}
\geometry{left=3.18cm,right=3.18cm,top=2.54cm,bottom=2.54cm}
\renewcommand{\today}{\number\year 年 \number\month 月 \number\day 日}
\renewcommand{\captionlabelfont}{\small}
\renewcommand{\captionfont}{\small}

\begin{document}

\begin{figure}
    \centering
    \includegraphics[width=8cm]{upc.png}
    \label{figupc}
\end{figure}

\begin{center}
	\quad \\
	\quad \\
	\heiti \fontsize{45}{17} \quad \quad \quad
	\vskip 1.5cm
	\heiti \zihao{2} 《计算科学导论》个人职业规划
\end{center}

\vskip 1.7cm

\begin{quotation}
	\doublespacing
    \zihao{4}\par\setlength\parindent{7em}
	\quad

	学生姓名:\underline{\quad \qquad \ 张森 \ \qquad \quad}

	学\hspace{0.6cm} 号:\underline{\qquad \ 1907010114 \ \qquad}
		
	专业班级:\underline{\qquad \ 计算1901 \ \qquad  }
		
    学\hspace{0.6cm} 院:\underline{计算机科学与技术学院}

	\vskip 2cm
	\centering
	\begin{table}[h]
        \centering
        \zihao{4}
        \begin{tabular}{|c|c|c|c|c|c|c|c|c|}
            \hline
            \multicolumn{5}{|c|}{分项评价} &\multicolumn{2}{c|}{整体评价}  & 总    分 & 评 阅 教 师\\
            \hline
            自我 & 环境 & 职业 & 实施 & 评估与 & 完整性 & 可行性 &\multirow{2}*{} &\multirow{2}*{}\\
            分析& 分析& 定位 & 方案 & 调整 & 20\% & 20\% & ~&~ \\\
            10\% & 10\% & 15\% & 15\% & 10\% & &  &~ &~\\
            \cline{1-7}
            & & & & & & & ~&~ \\
            & & & & & & & ~&~ \\
            \hline
        \end{tabular}
    \end{table}
    \vskip 1.7cm
    \today
\end{quotation}

\thispagestyle{empty}
\newpage

\setcounter{page}{1}

\section{自我分析}

\subsection{自然条件}

\begin{itemize}
    \item 性别:男
    \item 年龄:18
    \item 身体条件:良好
    \item 健康状况:良好,但缺乏锻炼
    \item 居住城市:家乡为诸城,现居青岛
\end{itemize}

\subsection{性格分析}

通过一些性格测评,再结合自身的自我分析、同学们的客观评价等,总结出我的个性性格:

我充满责任心,善于自我控制和自我约束,重视自己的义务和承诺,即使遇到了很大的障碍或者诱惑也能够坚持完成自己的本职工作。而且对待事务常常有自己的计划,喜欢按部就班的完成。这与自然界的“蜜蜂”十分相似。

我的主动性较强,较少受到环境的制约,更倾向于主动对环境做出创造和改变。在面临不利条件和缺陷时能迎刃而上,努力化解困难克服危机,常常成为传达组织使命、发现解决问题的先导者。

但是,我对事情的考虑不够周密,怯于在公共场合发表自己的看法或者意见,对待一些决策性事务的时候往往犹豫不决。这些性格往往会对我的日常学习和人际交往产生一定的影响。

\subsection{教育与学习经历}

在小学和初中阶段,我接触到了计算机,在母亲的“玩游戏是玩他人的思想,真正自己设计出游戏,让别人玩你的思想才是真的厉害”的勉励下,我了解了Java的基本语法,进而与程序设计竞赛搭建起了桥梁。

在高中和大学阶段,我参加了多次有关程序设计竞赛的算法培训,有一定的算法与程序设计基础,下表列举了迄今为止的部分程序设计赛事的获奖情况,部分奖项可作为求职的“加分项”。

\begin{tabular}{|c|c|c|}
    \hline
    赛事名称 & 奖项 & 获奖时间 \\
    \hline
    全国青少年信息学奥林匹克竞赛联赛 & 二等奖 & 2016.11 \\
    \hline
    全国青少年信息学奥林匹克竞赛联赛 & 一等奖 & 2017.11 \\
    \hline
    全国青少年信息学奥林匹克竞赛冬令营 & 铜牌 & 2018.2 \\
    \hline
    亚洲与太平洋地区信息学奥林匹克竞赛 & 铜牌 & 2018.5 \\
    \hline
    第44届ACM-ICPC国际大学生程序设计竞赛亚洲区域赛(南京站) & 银牌 & 2019.11 \\
    \hline
    第44届ACM-ICPC国际大学生程序设计竞赛亚洲区域赛(沈阳站) & 铜牌 & 2019.12 \\
    \hline
    2019年“端点科技杯”中国石油大学(华东)新生程序设计竞赛 & 冠军 & 2019.12 \\
    \hline
\end{tabular}

综合分析,我偏向于对自然科学的研究,而不喜对人文社科的研究,这与我的自然条件、性格分析等方面有关。

\subsection{工作与社会阅历}

我参加了一些社会实践活动。曾帮助某公司分发传单,作为某高校程序设计竞赛的志愿者参与志愿工作,帮助其他同学疏导心理问题等。这些活动帮助我提升自己的人际交往能力,克服在人际交往上的障碍。

我还参与了牛客网(nowcoder.com)的部分比赛的出题、翻译工作,并且有偿为牛客网校招提供试题。出题活动极大的拓宽了我的视野,帮助我认识到了同一领域的其他大牛,结交了不少朋友。

\subsection{知识、技能与经验}

我对C++有着较深的理解,此外还接触了Python、Java、HTML、CSS、JavaScript等多种编程语言,并且正在学习Flask框架和Bootstrap框架。

我懂得使用各种办公工具(如Office、Latex、Markdown等),会使用Github、CSDN、知乎等查阅资料,能在中国知网、校图书馆等搜集文献和书籍资料。我有较强的自主学习能力,能主动学习自己想学习的知识。

我还有算法与程序设计及其相关工作的经验,这里不再赘述。


\subsection{兴趣爱好与特长}

\begin{itemize}
    \item 学习过吉他,并获得了7级证书。
    \item 喜欢听纯音乐和英文音乐
    \item 会下象棋、军旗,喜欢一些博奕类游戏
\end{itemize}

\section{环境分析}

\subsection{社会环境分析}

当今世界形势变化莫测,暗流涌动,但我非常幸运的生活在祖国母亲的怀抱中,享受着和平安宁的生活。中美贸易摩擦愈演愈烈,芯片领域危机四伏,我国在高新技术领域仍有较大短板,受制于人。

同时,我国综合国力飞速提升。经济实力上,我国的GDP总值早已跃居世界第二位;国防实力上,“辽宁舰”“山东舰”航母编队整装待发;科技实力上,研究所的研究人员们攻克了一道道技术难关,打破了一个个技术壁垒……我国正大步迈向社会主义现代化强国,这为我未来的发展提供了极为稳定的环境和巨大的机遇。

\subsection{家庭环境分析}

我所生活的家庭温馨和睦,父母均为一线工人,家庭有稳定的经济收入,但需要赡养三位老人,偿还购房贷款,维持正常生活等。家人期望我能够对计算机科学方面做出一定贡献,偏向于从事与计算机应用相关的工作。但父母尊重我的发展选择,尊重我个人的职业意愿。

\subsection{职业环境分析}

计算机领域的飞速发展必然会带来网络安全方面的问题,未来,网络安全工程师必将在复杂的网络环境中扮演着必不可少的角色。

网络安全工程师主要负责产品的安全测试等活动,负责建立安全基线和安全测试用例库,以便制程全球市场项目。他需要对网络规划方面有较深的理解,熟悉常见的安全攻防技术和安全漏洞,能够根据测试需要构建个性化的安全测试工具等。任何一家大型互联网公司必然会重视网络方面的安全,具有可观的发展前景。

\subsection{地域与人际环境分析}

北京市是我国的政治、文化、科技创新、国际交往中心,同时也是国家中心城市、超大城市和现代化国际城市,是我国四个直辖市之一。

北京位于华北地区,地势西北高、东南低,三面环山,气候主要以北温带半湿润大陆性季风气候为主,夏季高温多雨,冬季寒冷干燥,春秋时期较为短促。它一座有着三千多年历史的古都,拥有着悠久的历史文化。

作为国家的首都,北京将主要承担政治中心的工作,同时兼顾其他领域的发展。北京是全国教育最发达的地区,拥有世界第三大图书馆中国国家图书馆,聚集了全国数量最多的重点高校,具有雄厚的科技实力和巨大的发展前景。

且北京云集许多互联网公司总部和研发中心,就业岗位多,诸多领域的大牛均生活在这座城市,人际环境良好,但存在房租食宿贵、交通较为拥堵等问题。

\section{职业定位}

\subsection{行业领域定位与理由}

基于上述的了解与分析,我认为我适合网络安全相关的工作,理由如下:

第一,我对网络安全方面有浓厚的兴趣。在计算科学导论中,我了解到了网络安全的bloodzer0的安全知识体系,在新生研讨课中,我了解到了石乐义教授的研究方向和一些成果。这些见闻引起了我对网络安全浓厚的兴趣。

第二,网络安全工作是企业乃至国家必不可少的一部分。小到企业,需要工程师对网络服务进行安全评估和维护,大到国家,需要科研人员提高我国应对敌对势力的网络攻击的能力。

\subsection{职业岗位起点定位与理由}

我认为,在学业结束(考虑读研)后,我会成为一名企业的网络安全工程师,理由如下:

第一,在读研期间我将进行对网络安全知识体系的学习。这有利于我更好的融入工作,也给予我更多的时间去将相关知识研究透彻。

第二,网络安全工程师偏向于实践,是上升到理论研究层面的“跳板”。在企业的工作经验能让我更快的熟练运用所学知识,提升自己的实践能力,对理论知识有更加深入的理解,进而进行一定的理论研究。

\subsection{职业目标与可行性分析}

\subsubsection{短期目标}

完成学业,力争校内保研资格,对网络安全方面有更加深入的理解,从事一定的兼职工作缓解家庭负担,获得研究网络安全、进行简单网络攻防的技能。

\subsubsection{中长期目标}

中期目标成为一名网络安全工程师,在企业从事网络安全相关方面的工作,为企业的网络服务安全提供技术支持,谋取利益。

长期目标成为一名网络安全方面的科研人员,在相关机构从事网络安全的科学研究,为国家的网络安全做出力所能及的贡献。

\section{实施方案}

\subsection{程序设计竞赛}

通过程序设计竞赛及其日常训练来不断提升自己的算法能力和编程能力,具体方案如下:

\begin{itemize}
    \item 每月学习至少一项新算法,并进行一定练习至掌握程度
    \item 积极参加ACM训练,在不断练习中查漏补缺,提升算法熟练度
    \item 提升对思维题的思考与探索能力,弥补自身考虑不全面的缺点
    \item 通过程序设计竞赛结交竞赛圈和与之相关企业圈的伙伴与朋友
\end{itemize}

\subsection{日常生活学习}

通过计划来提升日常学习生活质量,增强自己的人际交往能力进而发展人脉,缓解并释放工作压力,保证自己的身心健康,具体方案如下:

\begin{itemize}
    \item 认真学习每一门课程,重视每一项课程作业和考试考核
    \item 积极参加“石光”活动,发展人脉关系,释放学习压力
    \item 积极参加体育锻炼,保证身心健康
\end{itemize}

\subsection{其他}

根据实际情况在评估与调整阶段动态调整每一学期、每一学年的计划,更加纵容的向着实现个人职业规划的目标砥砺奋斗,不断前行。

\section{评估与调整}

\subsection{评估时间}

毕业前,每学期进行一次简要的评估与调整,适当调整下学期计划,每学年进行一次较为详细的评估与调整,适当调整下学年计划。

毕业后,每半年进行一次评估与调整,适当制定符合实际的月度计划和季度计划,并根据实际情况和期望目标的差异进行合理分析、适当调整。

\subsection{评估内容}

\begin{itemize}
    \item 评估是否完成了阶段性的计划和目标,量化成果与期望值
    \item 评估对比日常生活开销、补贴家用等必要消费与收入是否达到理想状态
    \item 评估自己的科研能力和技术能力是否较以前有所进步,是否学习到新的知识,是否能应用技术为人民、国家、社会做出更大的贡献等
    \item 是否拥有一个小团队钻研某一难题或开发某一项目
    \item 随时间的发展评估评估内容的合理性和完整性,进而改善评估内容
\end{itemize}

\subsection{调整原则}

根据自身情况与环境变化进行合理调整,需要考虑具体实施方案的可行性和目标可达性,进而分步设立目标。

\end{document}
